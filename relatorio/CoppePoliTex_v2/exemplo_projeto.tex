%%
%% This is file `example.tex',
%% generated with the docstrip utility.
%%
%% The original source files were:
%%
%% coppe.dtx  (with options: `example')
%% 
%% This is a sample monograph which illustrates the use of `coppe' document
%% class and `coppe-unsrt' BibTeX style.
%% 
%% \CheckSum{1416}
%% \CharacterTable
%%  {Upper-case    \A\B\C\D\E\F\G\H\I\J\K\L\M\N\O\P\Q\R\S\T\U\V\W\X\Y\Z
%%   Lower-case    \a\b\c\d\e\f\g\h\i\j\k\l\m\n\o\p\q\r\s\t\u\v\w\x\y\z
%%   Digits        \0\1\2\3\4\5\6\7\8\9
%%   Exclamation   \!     Double quote  \"     Hash (number) \#
%%   Dollar        \$     Percent       \%     Ampersand     \&
%%   Acute accent  \'     Left paren    \(     Right paren   \)
%%   Asterisk      \*     Plus          \+     Comma         \,
%%   Minus         \-     Point         \.     Solidus       \/
%%   Colon         \:     Semicolon     \;     Less than     \<
%%   Equals        \=     Greater than  \>     Question mark \?
%%   Commercial at \@     Left bracket  \[     Backslash     \\
%%   Right bracket \]     Circumflex    \^     Underscore    \_
%%   Grave accent  \`     Left brace    \{     Vertical bar  \|
%%   Right brace   \}     Tilde         \~}
%%
\documentclass[grad,numbers]{coppe}
\usepackage{amsmath,amssymb}
\usepackage{hyperref}
\usepackage[utf8]{inputenc}
\usepackage[brazil]{babel}
\usepackage[T1]{fontenc}
\usepackage{graphicx}

\makelosymbols
\makeloabbreviations

\begin{document}
  \title{IDENTIFICAÇÃO FOTOMÉTRICA DE SUPERNOVAS ATRAVÉS DE ALGORITMOS DE MACHINE LEARNING}
  \foreigntitle{SUPERNOVA PHOTOMETRIC IDENTIFICATION USING MACHINE LEARNING ALGORYTHMS}
  \author{Felipe Matheus}{Fernandes de Oliveira}
  \advisor{Prof.}{Amit}{Bhaya}{D.Sc.}
  \advisor{Prof.}{Ribamar}{Rondon de Rezende dos Reis}{Ph.D.}

  \examiner{Prof.}{Amit Bhaya}{D.Sc.}
  \examiner{Prof.}{Ribamar Rondon de Rezende dos Reis}{Ph.D.}
  \examiner{Prof.}{Heraldo Luís Silveira de Almeida}{D.Sc.}
  \examiner{Prof.}{Nome do Segundo Examinador Sobrenome}{D.Sc.}

  
  
  
  \department{ECA}% Confira a tabela a seguir para saber como preencher o comando \department de acordo com seu curso (Graduação - Poli) ou programa (Pós-Graduação - COPPE).
  
  %%%%%% Para alunos da POLI %%%%%%
  
  %% Course											Option
  %% Engenharia Ambiental                             EA
  %% Engenharia Civil                                 ECV
  %% Engenharia de Computação e Informação            ECI
  %% Engenharia de Controle e Automação               ECA
  %% Engenharia de Materiais                          EMAT
  %% Engenharia de Petróleo                           EPT
  %% Engenharia de Produção                           EPR
  %% Engenharia Eletrônica e de Computação            EEC
  %% Engenharia Elétrica                              EET
  %% Engenharia Mecânica                              EMC
  %% Engenharia Metalúrgica                           EMET
  %% Engenharia Naval e Oceânica                      ENO
  %% Engenharia Nuclear                               ENU
  
  
  %%%%%% Para alunos da COPPE %%%%%%
  
  %% Program											Option
  %% Engenharia Biomédica								PEB
  %% Engenharia Civil									PEC
  %% Engenharia Elétrica								PEE
  %% Engenharia Mecânica								PEM
  %% Engenharia Metalúrgica e de Materiais				PEMM
  %% Engenharia Nuclear									PEN
  %% Engenharia Oceânica								PENO
  %% Planejamento Energético							PPE
  %% Engenharia de Produção								PEP
  %% Engenharia Química									PEQ
  %% Engenharia de Sistemas e Computação				PESC
  %% Engenharia de Transportes							PET
  
  
  
  
  
  
  \date{07}{2019}

  \keyword{Machine Learning}
  \keyword{Gaussian Process Fitting}
  \keyword{Supernova Photometric Identification}

  \maketitle

  \frontmatter
  
  \makecatalog
  
  \dedication{A alguém cujo valor é digno desta dedicatória. A você Goku}

  \chapter*{Agradecimentos}

  Gostaria de agradecer a todos os Sayajins por terem nos salvado das garras maléficas de Freeza, Cell e Majin Boo.

  \begin{abstract}
%
%  Com a finalidade de estudar a expansão do universo, a cosmologia busca identificar  as curvas de luz de objetos astronômicos que sejam referentes a supernovas do tipo IA. Com o grande aumento do número de amostras de objetos astronômicos, o método usado para a identificação (espectroscopia) não consegue realizar uma quantidade significativa de classificação devido ao seu alto custo. Entretanto, já possuindo uma classificação precisa para um grande número de dados, podemos usar algoritmos de Machine Learning para classificar esse grande número de supernovas através de uma maneira menos custosa, esse algoritmo é a classificação fotométrica.

Com a finalidade de estudar a expansão do universo, a cosmologia busca clasificar diferentes tipos de objetos astronômicos. Entretanto, com o crescente aumento do número de objetos detectados, o método normalmente usado para a classificação (espetroscopia) torna-se muito custoso. Como consequência, utiliza-se um método com baixo custo (fotometria) embasado em algoritmos de aprendizado de máquina para a classificação desse vasto número de dados. Nesse contexto, o presente trabalho estuda otimizações para a melhoria desses algoritmos de aprendizado de máquina.
  \end{abstract}

  \begin{foreignabstract}

  To explore the expansion history of the universe, cosmology classifies different types of astronomic objects using spectroscopy. With their sample sizes increasing, spectroscopy methods cannot handle this amount of data. As a solution to this issue, photometric identification is crucial to fully exploit these large samples due to its simplicity. Once photometric identification uses machine learning algorithms, the following work tries to optimize those algorithms.

  \end{foreignabstract}

  \tableofcontents
  \listoffigures
  \listoftables
  \printlosymbols
  \printloabbreviations

  \mainmatter
%  \doublespacing

  \chapter{Introdução}

	\section{Tema e Contextualização}
	

	Dentro da cosmologia, existe a necessidade de determinar distâncias luminosas \ref{?} para modelar seus estudos, e como ferramenta dessa tarefa, utilizam-se as curvas de luz provenientes de supernovas do tipo IA.
	
	No passado, o conjunto de dados de supernovas era pequeno o suficiente para poder analisar a maior parte dos objetos usando o método da espectroscopia, um método lento e custoso que, no entanto, confirma precisamente o tipo de cada uma delas.
	
	Contudo, com o avanço das pesquisas e da tecnologia de telescópios, a astronomia está entrando em uma era de conjuntos massivos de dados, tornando-se necessário a adoção de técnicas automatizadas mais simples e práticas para classificar a enorme quantidade de objetos astronômicos captados, pois, através da espectroscopia não seria possível.
	 
	Nesse contexto, foram desenvolvidas diferentes abordagens para levantar essa grande quantidade de objetos captados através de um método chamado fotometria. Dentre essas abordagens, várias utilizam aprendizado de máquina.
	
	Por fim, a ideia do projeto foi derivada de um artigo publicado pela cientista \citet{lochner}. A autora busca criar uma maneira automática de classificação fotométrica usando as curvas de luz obtidas através da fotometria, essas que já foram devidamente classificadas no passado utilizando espectroscopia. Caracterizando assim um problema de classificação, visto que temos os tipos de supernovas dados pela espectroscopia, e suas representações fotométricas para que possamos criar e treinar modelos.
	
	\section{Problemática}
	
	Tendo em vista que o artigo \cite{lochner} testou e validou diversos \textit{pipelines}, neste trabalho adotamos aquele que ela obteve o melhor resultado de classificação, e a partir desse ponto, aplicamos modificações e avaliamos quaisquer possíveis melhoras.
	
	O \textit{pipeline} escolhido é constituído majoritariamente de 3 partes:
	
	\begin{itemize}
		\item Processo Gaussiano
		\item Transformada de Wavelet
		\item Floresta Aleatória
	\end{itemize}
	
	A problemática encontra-se no método de interpolação chamado \textbf{Processo Gaussiano}. Por ser um método de interpolação, espera-se que o mesmo defina uma função que passe pelos pontos obtidos respeitando suas incertezas. Entretanto, em alguns casos o gráfico interpolado é uma constante \ref{fig:ExReto}. O que não possui sentido físico, visto que se trata do fluxo de luz após uma explosão de um objeto astronômico.
	
	Através do Instituto de Física da UFRJ surgiu a ideia principal deste projeto que é a de tentar corrigir essas interpolações que apresentam comportamento constante durante toda a observação.
	
	Em paralelo, outras ideias foram aplicadas na tentativa de obter uma melhora no algoritmo original da \textit{Lochner}.
	
	\begin{figure}[ht]
		\centering
		\includegraphics[width=15cm]{Ex_reta.png}
		\caption{Exemplo da interpolação referente a um gráfico de fluxo de luz.}
		\label{fig:ExReto}
	\end{figure}
	
	\section{Delimitação}
	
	Todos os dados utilizados tanto neste trabalho quanto no artigo \cite{lochner}, foram extraídos da base de dados oferecida pelo desafio \citet{challenge}. A base de dados é de domínio público e pode ser obtida no {LINK DO DOWNLOAD}.
	
	Devido a granularidade dos dados brutos, é necessário um pré-tratamento visando buscar apenas os dados que iremos utilizar. Esse pré-tratamento foi aproveitado do \textit{pipeline} do Instituto de Física da UFRJ, cujo processo visa ler os arquivos de texto transformando cada informação em uma chave de dicionário. 
	
	Ao fim do uso da seleção de dados citada acima, obtivemos valores em forma de dicionários em Python para cada objeto astronômico. Esses dicionários serão os \textbf{Dados Brutos} no escopo deste trabalho.   
	
	\section{Objetivos}
	
	O objetivo do trabalho é otimizar determinados pontos do \textit{pipeline} utilizado atualmente, após um estudo inicial ter sido feito para identificar pontos frágeis ou passíveis de melhoras.
	
	Esses pontos são apresentados e justificados a seguir.
	
		\subsection{Processo Gaussiano \textit{Gaussian Process}}
		
		O Processo Gaussiano (\textit{Gaussian Process} ou \textbf{GP}), é a primeira parte do tratamento de dados após a transformação dos dados brutos em dicionários de Python.
		
		Cada objeto astronômico irá possuir uma quantidade de pontos que representam a intensidade do fluxo de luz captado em determinado dia, bem como a incerteza desse valor. Assim, busca-se fazer uma interpolação para obter um gráfico que passe por determinados pontos. 
		
		O objetivo ao abordar esse ponto é consertar interpolações que não condizem com a interpretação física, como pode ser vista na figura \ref{fig:ExReto}.
		
		\subsection{Aprendizagem Profunda}		

		\subsection{Tratamento de \textit{Outliers}}		

	\section{Metodologia}
	
		\subsection{Desenvolvimento}
		
		\subsection{Avaliação dos Resultados}
		
	\section{Descrição}
	Dizer aqui o que vai ter nos capítulos a frente
	
  \chapter{Fundamentos Teóricos}
  
 	\section{Processo Gaussiano \textit{Gaussian Process}}
 	
 	\section{Rede Neural Convolucional}
  
 	\section{Demais Conceitos}
 	
 		Conceitos menos usados ao longo do \textit{pipeline}, estão brevemente explicados a seguir.
 		
 		\subsection{Análise de Componentes Principais}
 		
 		\subsection{Transformadas de \textit{Wavelet}}
 		
  		\subsection{Validação Cruzada}

  \chapter{Modelo Atual}  		

	\section{Pré-processamento dos Dados Brutos}
	
	\section{\textit{Pipeline Atual}}

  \chapter{Tratamento de \textit{Outliers}}
 	
 	\section{Estratégias Utilizadas}  	
 		
 	\section{Resultados e comparações}
 		
  \chapter{Rede Neural Convolucional}
 	A ideia foi blablabla usar deep learning, vimos exemplos das roupas, e colocamos la
 	deixamos a princípio o pipeline do Marcelo
 	 	
 	 \section{Geração de Imagens e Parâmetros}  	
 		Os parâmetros na hora de gerar as imagens e o PQ
 	\section{Escolhas de Redes Neurais}  	
 		escolhemos 2CNN pra fazer por isso e por aquilo
 	
 	\section{Resultados e comparações}

  \chapter{Interpolação através de Processo Gaussiano}
  
	\section{Escolha da Biblioteca}
		
	\section{Aleatoriedades e \textit{Random Seeds}}
			
	\section{\textit{Kernel Functions}}
			
	\section{Demais Observações}
			Valor negativo e Mean function, erro grande, melhor overfittado que underfitado
			
	\section{Resultados das Interpolações}
		
	\section{Identificação e Justificativa do Erro}
 
  \chapter{Resultados e Discussões}

  \chapter{Conclusões}
 
	\section{Conclusões Finais} 	  
		
			Falar que o Algoritmo de hoje já é bem robusto
		
	\section{Trabalhos Futuros} 
		
			Falar de consertar o Erro e pah
		
  \chapter{Introdução}

  Segundo a norma de formatação de teses e dissertações do
  Instituto Alberto Luiz Coimbra de Pós-graduação e Pesquisa de
  Engenharia (COPPE), toda abreviatura deve ser definida antes de
  utilizada.\abbrev{COPPE}{Instituto Alberto Luiz Coimbra de Pós-graduação e Pesquisa de Engenharia}

  Do mesmo modo, é imprescindível definir os símbolos, tal como o
  conjunto dos números reais $\mathbb{R}$ e o conjunto vazio $\emptyset$.
  \symbl{$\mathbb{R}$}{Conjunto dos números reais}
  \symbl{$\emptyset$}{Conjunto vazio}
  
  Você deve selecionar seu curso de engenharia usando o comando \texttt{\textbackslash department\{Sigla\}} e no lugar de Sigla inserir a sigla referente ao seu curso de engenharia. A tabela \ref{tab:courses} relaciona as siglas dos cursos de engenharia da Escola Politécnica da Universidade Federal do Rio de Janeiro (POLI-UFRJ), enquanto que a tabela \ref{tab:programs} relaciona as siglas dos programas de pós graduação da COPPE.\abbrev{POLI-UFRJ}{Escola Politécnica da Universidade Federal do Rio de Janeiro}
  
  
  \begin{table}[h]
    \caption{Siglas dos cursos de engenharia da Escola Politécnica da UFRJ.}
    \label{tab:courses}
    \centering
    {\footnotesize
    \begin{tabular}{|c|c|}
      \hline
      Sigla & Curso\\
      \hline
      EA &  Engenharia Ambiental \\
      ECV & Engenharia Civil\\
      ECI & Engenharia de Computação e Informação \\
      ECA & Engenharia de Controle e Automação \\
      EMAT & Engenharia de Materiais\\
      EPT & Engenharia de Petróleo\\
      EPR & Engenharia de Produção\\
      EEC & Engenharia Eletrônica e de Computação\\
      EET & Engenharia Elétrica\\
      EMC & Engenharia Mecânica\\
      EMET & Engenharia Metalúrgica\\
      ENO & Engenharia Naval e Oceânica\\
      ENU & Engenharia Nuclear\\
      \hline
    \end{tabular}}
    \end{table}
    
    
  \begin{table}[h]
  	\caption{Siglas dos programas de pós graduação da COPPE.}
  	\label{tab:programs}
  	\centering
  	{\footnotesize
  	\begin{tabular}{|c|c|}
  		\hline
  		Sigla & Curso\\
  		\hline
  		PEB & Engenharia Biomédica \\
  		PEC & Engenharia Civil\\
  		PEE & Engenharia Elétrica \\
  		PEM & Engenharia Mecânica \\
  		PEMM & Engenharia Metalúrgica e de Materiais\\
  		PEN & Engenharia Nuclear\\
  		PENO & Engenharia Oceânica\\
  		PPE & Planejamento Energético\\
  		PEP & Engenharia de Produção\\
  		PEQ & Engenharia Química\\
  		PESC & Engenharia de Sistemas e Computação\\
  		PET & Engenharia de Transportes\\
  		\hline
  	\end{tabular}}
  \end{table}


  Note também que todas as figuras ou tabelas devem ser citadas no texto. Como ocorre com as tabelas \ref{tab:courses} e \ref{tab:programs}. Para ilustrar o uso de figuras em \LaTeX, considere as figuras \ref{fig:poli} e \ref{fig:coppe}.
  
   \begin{figure}
      \centering
      \includegraphics[width=5cm]{poli-logo.pdf}
      \caption{Logotipo da POLI-UFRJ.}
      \label{fig:poli}
    \end{figure}
    
    \begin{figure}
       \centering
       \includegraphics[width=5cm]{coppe-logo.pdf}
       \caption{Logotipo da COPPE.}
       \label{fig:coppe}
     \end{figure}

  \chapter{Revisão Bibliográfica}

  Para ilustrar a completa adesão ao estilo de citações e listagem de
  referências bibliográficas, a Tabela \ref{tab:citation} apresenta citações de alguns dos trabalhos contidos na norma fornecida pela CPGP da
  COPPE, utilizando o estilo numérico.

  \begin{table}[h]
  \caption{Exemplos de citações utilizando o comando padrão
    \texttt{\textbackslash cite} do \LaTeX\ e
    o comando \texttt{\textbackslash citet},
    fornecido pelo pacote \texttt{natbib}.}
  \label{tab:citation}
  \centering
  {\footnotesize
  \begin{tabular}{|c|c|c|}
    \hline
    Tipo da Publicação & \verb|\cite| & \verb|\citet|\\
    \hline
    Livro & \cite{book-example} & \citet{book-example}\\
    Artigo & \cite{article-example} & \citet{article-example}\\
    Relatório & \cite{techreport-example} & \citet{techreport-example}\\
    Relatório & \cite{techreport-exampleIn} & \citet{techreport-exampleIn}\\
    Anais de Congresso & \cite{inproceedings-example} &
      \citet{inproceedings-example}\\
    Séries & \cite{incollection-example} & \citet{incollection-example}\\
    Em Livro & \cite{inbook-example} & \citet{inbook-example}\\
    Dissertação de mestrado & \cite{mastersthesis-example} &
      \citet{mastersthesis-example}\\
    Tese de doutorado & \cite{phdthesis-example} & \citet{phdthesis-example}\\
    \hline
  \end{tabular}}
  \end{table}
  
  É importante notar que, segundo a \href{http://www.poli.ufrj.br/graduacao_projeto.php}{Norma para a Elaboração Gráfica do Projeto de Graduação} da Escola Politécnica da UFRJ para trabalhos de conclusão de curso de engenharia de julho de 2012, as referências bibliográficas podem ser apresentadas de duas formas: $(i)$ Referências numeradas e $(ii)$ Referências em ordem alfabética. Para exibição numerada, em que a exibição das referências bibliográficas segue a ordem de citação usada no texto, use o comando \texttt{\textbackslash bibliographystyle\{coppe-unsrt\}}. Para exibição de referências bibliográficas em ordem alfabética, basta usar o comando \texttt{\textbackslash bibliographystyle\{coppe-plain\}} ao final do documento. 
  
  
  \chapter{Método Proposto}
  
  
  
  \chapter{Resultados e Discussões}
  
  
  
  \chapter{Conclusões}
  
  
  
  

  \backmatter
  \bibliographystyle{coppe-unsrt}
  \bibliography{example}

  \appendix
  \chapter{Algumas Demonstrações}
\end{document}
%% 
%%
%% End of file `example.tex'.
